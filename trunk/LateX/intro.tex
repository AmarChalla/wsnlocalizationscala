%Now starts your actual paper. Only the first letter of the Introduction needs a larger, so-called
%drop letter. ALL OTHER SECTIONS/SUBSECTIONS START WITH A NORMAL LETTER.

% form to use if the first word consists of a single letter:
% \IEEEPARstart{A}{demo} file is ....
% 
% form to use if you need the single drop letter followed by
% normal text (unknown if ever used by IEEE):
% \IEEEPARstart{A}{}demo file is ....

% You must have at least 2 lines in the paragraph with the drop letter
% (should never be an issue)
% Here we have the typical use of a "W" for an initial drop letter
% and "ireless sensor networks" in caps to complete the first word.
\IEEEPARstart{A}{formal} definition of Wireless Sensor Networks is given in \cite{akyildiz2002wsn}. The purpose of a WSN is to monitor some physical phenomena such as the ambient temperature, air humidity, and the presence or absence of a certain chemical. Usually, this data is collected at a common point, the data sink, where the data can be further processed or analyzed by the user. WSNs enable a great deal of new applications including environmental and habitat monitoring, smart homes and battlefield control \cite{tolle2005mr} \cite{chong2003sne}.

Accurate and low-cost sensor localization is an essential service and is important to many of these applications. The measured data is generally meaningless without knowing where it originated from. That being said, there are other reasons to acquire the location of a sensor. Location can be a goal in itself, such as in warehousing and manufacturing applications. Another added benefit is the possibility of geographical based routing. 

Given the popularity of GPS, one would think of it as a possible solution. However, this is considered a bad solution because of several reasons. Firstly, a great deal of WSN applications are built  for indoor environments. Clearly, GPS does not suffice as it requires Line-of-Sight with at least four satellites. Secondly, GPS requires significant power to operate, which is a sparse resource in WSNs. Thirdly, GPS adds to the size and cost of a WSN node which should be kept as low as possible. This does not mean, however, that GPS is entirely out of the question. A small number of WSN nodes will need a priori knowledge about their location; this can be done by a GPS receiver or by manually mapping the node to a position on a map, or some other method. The number of these nodes should be kept as low as possible, in order to keep the deployment cost and time low. 

\begin{table}[ht]
\caption{List of abbreviations}
\centering
\begin{tabular}{l l} \hline
	WSN 	& 	Wireless Sensor Network \\
	WSNs  &   Wireless Sensor Networks \\
	RSS 	& 	Received Signal Strength \\
	RSSI  &   Received Signal Strength Indication \\
	TOA   &		Time Of Arrival \\
	AOA   &   Angle Of Arrival \\
	CL 		& 	Centroid Localization \\
	WCL 	& 	Weighted Centroid Localization \\
	ML		&		Maximum Likelihood \\
	LS 		& 	Least Square \\
	LOS   &   Line Of Sight \\
	AN 		& 	Anchor Node \\
	BN 		& 	Blind Node \\
	RN    &   Root Node \\
	GUIs  &   Graphical Unit Interfaces \\ \hline
\end{tabular}
\end{table}

Localization techniques which are specific for WSNs are based on pair-wise measurements between nodes to estimate the positions. A small fraction of the network should have a known position as described in  the previous paragraph. These nodes are called anchor nodes. The other kind of nodes, without a known position, are called blind nodes. Thus, the goal of a localization system is to determine the position of the blind nodes by communicating with the anchor nodes. We can divide these techniques into two categories: range-based and connectivity-based. Range-based methods estimate the distance between nodes with ranging method such as ToA, AoA and RSS. These techniques typically provide superior accuracy but are more complex than connectivity-based  algorithms. These do not estimate the distance between nodes but determine the position of a blind node by their proximity to anchor nodes. \cite{hightower2001lsu}[Wij]  describe agreement properties of localization techniques in more detail. 
There are other localization techniques which are used in other networks, the most common being RF fingerprinting. In this method, a map based on radio signals is created. In the first phase (offline-phase) we measure RSS on different points on our map and store them in a database. The second phase (real-time phase) measures the RSS of a blind node and compares it to reference points on the map constructed during the first phase, in order to determine the location of the blind node. The problem with this technique is that the RF pattern changes when the environment changes, which means that the radio map is no longer up to date and the accuracy will therefore drop dramatically. A second problem is that these maps are time-consuming to build. Other techniques exist, but they are far less common and unsuitable for WSNs.  These techniques will not be used in this paper. 

Although many ranging techniques exist, in this paper we have narrowed them down to RSS-based ranging. 
RSS-based ranging is founded on the principle that RSS attenuates with distance due to free-space losses and other factors. RSS is generally considered as a bad method because of its high variability due to interference, multipath and shading. Errors can be divided into two categories, environmental and device errors. Environmental errors are due to the wireless channel; these include multipath, shadowing effects and interference from other radio sources. Device errors are generally calibration errors. The most important consideration here is to keep the transmitted power constant, despite inter-device differences and depleting batteries. Environmental errors can also be divided into two parts: rapid time varying errors and static environment dependent errors. The first is due to movement of people, additive noise and interference. This can mostly be modeled as Gaussian noise. As a result, this can be reduced considerably by averaging multiple RSSI measurements [15]. The second type of error is due to the varying properties of the environment, such as multipath and shadowing. Since the layout of the environment and the placement of doors and furniture cannot be known without prior knowledge, this error should be modeled as random. 

In order to create an accurate localization system based on RSS, the wireless channel properties and these other degrading effects must be modeled as accurately as possible. All these factors seem to give RSSI measurements a large random factor, thus making it very unpredictable. Even with very good modeling, it is inevitable that errors remain present because of the random factor; thus any good localization algorithm should also account for these factors. The upside of using RSS as a ranging method is that the radio can be used for communication and localization. This makes RSS very interesting because there is no need for additional hardware. Other ranging methods such as TOA, especially combined with ultrasound, usually yield better results, but require additional specialized hardware which adds to the size and cost of a node.

RSS-based localization can be divided into three categories:
\begin{itemize}
	\item Range-based or fine-grained localization 
	\item Connectivity-based or coarse-grained localization 
	\item RF Fingerprinting, as described in the previous paragraph
\end{itemize}
We will restrict our algorithms to the first two types to satisfy the ad-hoc requirement of the network. 

As WSNs have some unique properties, the algorithms in this paper have been developed or selected with the following goals in mind: 
\begin{itemize}
	\item RSS-based : using this technique no additional hardware is required, thus the cost of a node can be kept low. However, because nodes have an antenna embedded on the PCB, it would be better to have an external antenna as these have a more uniform radiation pattern. 
	\item Distributed and self-organizing: The algorithm should be able to run locally on the nodes to avoid a central processing dependency. This is especially important for WSNs due to the fact that individual nodes and links between nodes are more prone to failure than in a traditional computing environment. Batteries may be depleted and radio links can be obscured.  
	\item Robust: The algorithms should account for localization errors and node failures. 
	\item Receiver-based: The task of localization is up to the blind node so that the network scales well. 
	\item Responsiveness: The localization latency needs to be kept as low as possible. Mobility is fairly limited in WSNs as most nodes have a static position; however, certain nodes can be mobile, so this factor needs to be accounted for as well. 
	\item Energy usage: Given the sparse amount available, processing and communication needs to limited. Unfortunately, this means that certain applications are not suitable for WSNs because of the high computational requirements and the lightweight microcontroller that drives the nodes. Communication between nodes needs to be limited as well because the radio requires much more power than the microcontroller. 
	\item Adaptive: We want our algorithm to be adaptive to the number of ANs and the density of the network. If the density or the number of ANs rises, the accuracy should improve. The algorithm should thus benefit from the high density of the WSN. The algorithm should still perform well with a low network density and AN ratio. 
	\item Multihop localization: Nodes not in range of an AN should still be able to localize themselves by the use of other BNs, this is referred to as cooperative or multihop localization \cite{patwari2005lnc}, compared to single-hop localization, where only blind nodes in range of enough anchor nodes can determine their position. 
\end{itemize}

The main contributions of this paper are: 
\begin{itemize}
	\item We present an overview of the existing algorithms and literature. 
	\item We compare three algorithms with quantitative measurements. 
	\item We researched the influence of the orientation of a node compared to a node with an external antenna.
	\item We present Senseless, a software framework to manage WSN localization.
	\item We interface this framework to SCALA, a localization middleware project. 
\end{itemize}

The rest of the paper will be organized as follows: chapter two summarizes related work, and provides an overview of the suitable algorithms and work that contribute to building a good algorithm; chapter three presents our software framework; chapter four presents the ranging and the various algorithms that we have implemented and tested framework; in chapter five the methods and results are presented and we conclude in chapter six. 